
% SCG paper templates are in the LatexTemplates folder of the following repo:
%	git clone git@scg.unibe.ch:scgbib
% read-only: 
%	git clone git://scg.unibe.ch/scgbib

\documentclass[12pt,a4paper]{article}
% \usepackage[mac]{inputenc} % Only useful on macs
% \def\baselinestretch{1.4} % uncomment for doublespacing
\usepackage{a4wide}
\usepackage{xspace}
\usepackage{graphicx}
% \graphicspath{{./figures/}}
% ============================================================
%% Uncomment the next few lines to get sf url links:
%\usepackage{url}            
%\makeatletter
%\def\url@leostyle{%
%  \@ifundefined{selectfont}{\def\UrlFont{\sf}}{\def\UrlFont{\small\sffamily}}}
%\makeatother
%\urlstyle{leo} % Now actually use the newly defined style.
%% Choose coloured or b/w links:
\usepackage[pdftex,colorlinks=true,pdfstartview=FitV,
 linkcolor=black,citecolor=black,urlcolor=black]{hyperref}
% \usepackage{hyperref}
\usepackage{needspace}
\newcommand{\needlines}[1]{\Needspace{#1\baselineskip}}
\usepackage{paralist}
% ============================================================
%:Markup macros for proof-reading
\usepackage{ifthen}
\usepackage[normalem]{ulem} % for \sout
\usepackage{xcolor}
\newcommand{\ra}{$\rightarrow$}
\newboolean{showedits}
\setboolean{showedits}{true} % toggle to show or hide edits
%\setboolean{showedits}{false} % toggle to show or hide edits
\ifthenelse{\boolean{showedits}}
{
	\newcommand{\meh}[1]{\textcolor{red}{\uwave{#1}}} % please rephrase
	\newcommand{\ins}[1]{\textcolor{blue}{\uline{#1}}} % please insert
	\newcommand{\del}[1]{\textcolor{red}{\sout{#1}}} % please delete
	\newcommand{\chg}[2]{\textcolor{red}{\sout{#1}}{\ra}\textcolor{blue}{\uline{#2}}} % please change
	\newcommand{\nbe}[3]{
		{\colorbox{#3}{\bfseries\sffamily\scriptsize\textcolor{white}{#1}}}
		{\textcolor{#3}{\sf\small$\blacktriangleright$\textit{#2}$\blacktriangleleft$}}}
}{
	\newcommand{\meh}[1]{#1} % please rephrase
	\newcommand{\ins}[1]{#1} % please insert
	\newcommand{\del}[1]{} % please delete
	\newcommand{\chg}[2]{#2}
	\newcommand{\nbe}[3]{}
}
%
\newcommand\rA[1]{\nbe{Reviewer A}{#1}{cyan}}
\newcommand\rB[1]{\nbe{Reviewer B}{#1}{olive}}
\newcommand\rC[1]{\nbe{Reviewer C}{#1}{magenta}}
\newcommand\ANS[1]{\nbe{Response}{#1}{teal}}

\newcommand{\THE}{\ins{the}\xspace} % "the" missing
\newcommand{\A}{\ins{a}\xspace} % "a" missing
\newcommand{\s}{\ins{s}\xspace} % "s" missing
\newcommand{\COMMA}{\ins{,}\xspace} % "," missing
\newcommand{\THAT}{\chg{which}{that}\xspace} % use "that", not "which"

% ============================================================
%:Box comments/edits
\usepackage[most]{tcolorbox}
\ifthenelse{\boolean{showedits}}
{
  \newtcolorbox{inserted}{%
       title=Inserted text:,
       colframe=blue,colback=blue!5!white,
       breakable,
       leftrule=0mm, 
       bottomrule=0mm,
       rightrule=0mm,
       toprule=0mm,
       arc=0mm, outer arc=0mm,
       oversize
  }
  \newtcolorbox{deleted}{%
       title=Deleted text:,
       colframe=red,colback=red!5!white,
       breakable,
       leftrule=0mm, 
       bottomrule=0mm,
       rightrule=0mm,
       toprule=0mm,
       arc=0mm, outer arc=0mm,
       oversize
  }
  \newtcolorbox{refactored}{%
       % title=Heavily modifed/refactored text:,
       title=Rewritten text:,
       colframe=blue,colback=red!5!white,
       breakable,
       leftrule=0mm, 
       bottomrule=0mm,
       rightrule=0mm,
       toprule=0mm,
       arc=0mm, outer arc=0mm,
       oversize
  }
}{
  \newenvironment{inserted}{}{}
  %\newenvironment{deleted}{ \begin{comment} }{ \end{comment} }
  \let\deleted\comment
  \newenvironment{refactored}{}{} 
}
% ============================================================
%:Put edit comments in a really ugly standout display
%\usepackage{ifthen}
\usepackage{amssymb}
\newboolean{showcomments}
\setboolean{showcomments}{true}
%\setboolean{showcomments}{false}
\newcommand{\id}[1]{$-$Id: scgPaper.tex 32478 2010-04-29 09:11:32Z oscar $-$}
\newcommand{\yellowbox}[1]{\fcolorbox{gray}{yellow}{\bfseries\sffamily\scriptsize#1}}
\newcommand{\triangles}[1]{{\sf\small$\blacktriangleright$\textit{#1}$\blacktriangleleft$}}
\ifthenelse{\boolean{showcomments}}
%{\newcommand{\nb}[2]{{\yellowbox{#1}\triangles{#2}}}
{\newcommand{\nbc}[3]{
 {\colorbox{#3}{\bfseries\sffamily\scriptsize\textcolor{white}{#1}}}
 {\textcolor{#3}{\sf\small$\blacktriangleright$\textit{#2}$\blacktriangleleft$}}}
 \newcommand{\version}{\emph{\scriptsize\id}}}
{\newcommand{\nbc}[3]{}
 \newcommand{\version}{}}
\newcommand{\nb}[2]{\nbc{#1}{#2}{orange}}
\newcommand{\here}{\yellowbox{$\Rightarrow$ CONTINUE HERE $\Leftarrow$}}
\newcommand\rev[2]{\nb{TODO (rev #1)}{#2}} % reviewer comments
\newcommand\fix[1]{\nb{FIX}{#1}}
\newcommand\todo[1]{\nb{TO DO}{#1}}
\newcommand\on[1]{\nbc{ON}{#1}{olive}} % add more author macros here
%\newcommand\XXX[1]{\nbc{XXX}{#1}{blue}}
%\newcommand\XXX[1]{\nbc{XXX}{#1}{brown}}
%\newcommand\XXX[1]{\nbc{XXX}{#1}{cyan}}
%\newcommand\XXX[1]{\nbc{XXX}{#1}{darkgray}}
%\newcommand\XXX[1]{\nbc{XXX}{#1}{gray}}
%\newcommand\XXX[1]{\nbc{XXX}{#1}{magenta}}
%\newcommand\XXX[1]{\nbc{XXX}{#1}{olive}}
%\newcommand\XXX[1]{\nbc{XXX}{#1}{orange}}
%\newcommand\XXX[1]{\nbc{XXX}{#1}{purple}}
%\newcommand\XXX[1]{\nbc{XXX}{#1}{red}}
%\newcommand\XXX[1]{\nbc{XXX}{#1}{teal}}
%\newcommand\XXX[1]{\nbc{XXX}{#1}{violet}}
% ============================================================
\newcommand{\seclabel}[1]{\label{sec:#1}}
%\newcommand{\secref}[1]{Section~\ref{sec:#1}} <- use \autoref instead!
\newcommand{\figlabel}[1]{\label{fig:#1}}
%\newcommand{\figref}[1]{Figure~\ref{fig:#1}}
\newcommand{\tablabel}[1]{\label{tab:#1}}
%\newcommand{\tabref}[1]{Table~\ref{tab:#1}}
% ============================================================
\newcommand{\bs}{\symbol{'134}} % backslash
\newcommand{\us}{\symbol{'137}} % underscore
\newcommand{\ie}{\emph{i.e.},\xspace}
\newcommand{\eg}{\emph{e.g.},\xspace}
\newcommand{\etal}{\emph{et al.}\xspace}
\newcommand{\etc}{\emph{etc.}\xspace}
% ============================================================
%% Fill this in or delete it
%\pdfinfo{
%  /Title()
%  /Author()
%  /Subject()
%  /Keywords()
%}
% ============================================================
%=========================REFERENCES========================
%========makes multiple references within one \cite command to work with ulem package
% This is a workaround so that one may place multiple references within one \cite command inside an edit command
% for example, \ins{\cite{Leue17a, Leue17b}}
% boolean 'usemcite':
% when set to true will wrap each \cite command with an \mbox command only if ?showedits? or "showcomments" is set to true
% otherwise \cite will not be wrapped around.
% if 'usemcite' is set to false, nothing will happen, and \cite remains \cite
% Caveat: an \mbox command does not break across a line, so be sure to set 'usemcite' command to false when preparing the final version of a paper
\newboolean{usemcite}
%\setboolean{usemcite}{true}
\setboolean{usemcite}{false} % to use or not mcite command
\ifthenelse{\boolean{usemcite}}
{
	\ifthenelse{\boolean{showedits}}
	{	
		\let\oldcite\cite
		\newcommand{\mcite}[1]{\mbox{\oldcite{#1}}}
		\renewcommand*\cite[1]{\mcite{#1}}
	}
	{
		\ifthenelse{\boolean{showcomments}}
		{
			\let\oldcite\cite
			\newcommand{\mcite}[1]{\mbox{\oldcite{#1}}}
			\renewcommand*\cite[1]{\mcite{#1}}
		}
		{}
	}
}
{}
%===========================================================
\title{SCG Paper Template}
\author{The Author\\
Software Composition Group,
University of Bern, Switzerland\\
\url{http://scg.unibe.ch}}
\date{\today\\\version}
% ============================================================
\begin{document}
\maketitle
\begin{abstract}
\end{abstract}
% ============================================================
\section{Introduction}
\seclabel{intro}

NB:
toggle the boolean {\tt showedits} to show/hide insertions/deletions, reviewer comments and responses (\eg to generate version with responses);
toggle the boolean {\tt showcomments} to show/hide editorial comments.

\on{here is a comment}

%\begin{figure}[tb]
%\begin{center}
%\includegraphics[width=\columnwidth]{fig}
%\caption{Adding a thread-safe mixin layer.}
%\figlabel{safe}
%\end{center}
%\end{figure}

%\begin{figure*}[tb]
%\begin{center}
%\includegraphics[width=\pagewidth]{fig}
%\caption{Adding a thread-safe mixin layer.}
%\figlabel{safe}
%\end{center}
%\end{figure*}

Here is some \emph{emphasized} text.

Editing macros:
\begin{itemize}
\item \meh{Rephrase this text}.
\item \ins{Insert this text}.
\item \del{Delete this text}.
\item \chg{Change this}{to this}.
\end{itemize}

\begin{deleted}
Obsolete text
\end{deleted}

\begin{inserted}
New text
\end{inserted}

\begin{refactored}
Refactored text
\end{refactored}

\rA{You are wrong}
\ANS{We are right!}

Inline enum list.
\begin{inparaenum}[\itshape i\upshape)]
	\item professional developers from industry, and
	\item real-world software systems.
\end{inparaenum}    


%%%%%%%%%%%%%%%%%%%%%%%%%%%%%%%%%%%%%%%%%%%%%%%%%%%%%%%%%%%%
\section{section}
\seclabel{section}
% ==========================================================
\subsection{subsection}
\seclabel{subsection}
% ----------------------------------------------------------
\subsubsection{subsub}
\seclabel{subsub}
% ----------------------------------------------------------
\subsection*{Acknowledgments}
\fix{Adapt/select as appropriate}
We gratefully acknowledge the financial support of the Swiss National Science Foundation for the project
``Agile Software Assistance'' (SNSF project No. 200020-181973, Feb. 1, 2019 - April 30, 2022).
% CHOOSE will support any student once a year for 500 Fr to present an OO paper
% See: http://choose.s-i.ch/sponsorships
We also thank CHOOSE, the Swiss Group for Original and Outside-the-box Software Engineering of the Swiss Informatics Society, for its financial contribution to the presentation of this paper.
% ESUG offers 150 Euros per research paper up to 3 times per institution per year
% See: http://www.esug.org/Promotion/YourPublication
We also thank ESUG, the European Smalltalk User Group, for its financial contribution to the presentation of this paper.

% ``Agile Software Assessment'' (SNSF project No. 200020-144126/1, Jan 1, 2013 - Dec. 30, 2015).
% ``Synchronizing Models and Code" (SNF Project No.\ 200020-131827, Oct.\ 2010 - Sept.\ 2012).
% ``Bringing Models Closer to Code" (SNF Project No.\ 200020-121594, Oct.\ 2008 - Sept.\ 2010).
% Hasler Foundation for the project ``Enabling the evolution of J2EE applications through reverse engineering and quality assurance''
% ``Analyzing, Capturing and Taming Software Change" (SNF Project No.\ 200020-113342, Oct.\ 2006 - Sept.\ 2008).
%``A Unified Approach to Composition and Extensibility" (SNF Project No.\ 200020-105091/1, Oct.\ 2004 - Sept.\ 2006).
% ``RECAST: Evolution of Object-Oriented Applications" (SNF Project No. 620-066077, Sept. 2002 - Aug. 2006).
% ``Tools and Techniques for Decomposing and Composing Software" (SNF Project No. 2000-067855.02, Oct. 2002 - Sept. 2004).
% ``Meta-models and Tools for Evolution Towards Component Systems" (SNF Project No. 20-61655.00, Oct. 2000 - Sept. 2002),
% ``A Framework Approach to Composing Heterogeneous Applications", (SNF Project No. 20-53711.98, Oct. 1998 - Sept. 2000),
% ``Infrastructure For Software Component Frameworks" (SNF Project No. 2000-46947.96, Oct. 1996 - Sept. 1998),
% ``Composing Active Objects" (SNF Project No. 2140610.94, Oct. 1994 - Sept. 1996),

% ============================================================
% \bibliographystyle{alphaurl}
% \bibliography{/home/scg/scgbib/scg}
% \bibliography{scg}
\end{document}
% ============================================================

