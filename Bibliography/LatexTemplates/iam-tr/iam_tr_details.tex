% --------------------------------------------------------------------------------
% BITTE FOLGENDEN ABSCHNITT GEMAESS KOMMENTAREN ANPASSEN !!
% --------------------------------------------------------------------------------
\newcommand{\reptitle}{
	% Titel des Technischen Berichts, Deutsch oder Englisch
	% Dieser kann mehrere Zeilen lang sein
	Nonummy in suscipit ad vero consectetuer odio
	ipsum minim lobortis odio facilisi
}
% --------------------------------------------------------------------------------
\newcommand{\repnumber}{
	% IAM technischer Bericht-Nummer
	% bitte eine gueltige TR Nummer bei techreports@iam.unibe.ch verlangen
	IAM-08-003
}
% --------------------------------------------------------------------------------
\newcommand{\repdate}{
	%  Datum des technischen Berichts
	% (bitte in Deutschen Format angeben: Tag. Monat Jahr)
	26. Mai 2008
}
% --------------------------------------------------------------------------------
\newcommand{\repshortauthors}{
	% Kurzform der Autoren, nur die ersten 4-5 angeben, Vornamen abkuerzen 
	% und ggf. Liste mit "et al." ergaenzen
	E. Eleutherius, T. Metrophanes, D. Georgios, N. Epaphras, 
	H. Sophronius, et al.
}
% --------------------------------------------------------------------------------
\newcommand{\repallauthors}{
	% Komplette Liste der Autoren in langer Form
	% mit ausgeschriebenen Vornamen
	Erasmus Eleutherius, Timotheus Metrophanes, Diokles Georgios,
	Nikomachos Epaphras, Heron Sophronius, Charis Ambrosia, Timothea Euphemia
}
% --------------------------------------------------------------------------------
\newcommand{\repcategories}{
	% "Categories" und "Subject Descriptors" gemaess http://www.
	C.2.1 [Computer-Communication Networks]: Network Architecture and Design;
	C.2.2 [Computer-Communication Networks]: Network Protocols;
	C.2.3 [Computer-Communication Networks]: Network Operations;
	C.2.4 [Computer-Communication Networks]: Distributed Systems	
}
% --------------------------------------------------------------------------------
\newcommand{\repterms}{
	% "General Terms" gemaess http://www.
 	Design, Management, Measurement, Performance, Reliability, Security, Verification
}
% --------------------------------------------------------------------------------
\newcommand{\repkeywords}{
	% "Key Words" gemaess http://www.
	peer-to-peer, wireless mesh networks, wireless sensor networks, overlay multicast,
	network security, automata theory
}
% --------------------------------------------------------------------------------
\newcommand{\repsections}
{
	% Fuer jedes Kapitel eine eigene Datei in "_chapter" erstellen
	% dann jedes Kapitel mit der Kapitel-Dateinamen und -Titel einbinden
	\includechapter{chapter1}{Lorem Ipsum}
	\includechapter{chapter2}{Lobortis duis iriuredolor odio nisl}
	\includechapter{chapter3}{Eum ut vero eum illum sed ut sussipit exerci nostrud at}
	\includechapter{biblio}{References}	
}
% --------------------------------------------------------------------------------
